\documentclass[class=NCU_thesis, crop=false]{standalone}
\usepackage{mathrsfs,amsmath}

\begin{document}

\chapter{理論模擬}
本章為理論模擬,主要分成四個章節。第一節中會先介紹展頻技術的基本原理,說明如何以相位調製去實作直接序列展頻 (DSSS)。第二節會從數學原理出發,瞭解隨時間調變相位會如何改變頻率的分佈,再以單頻波之相位調製去模擬調製後的頻譜變化。第三節會將調製函數改為隨機訊號,模擬經隨機相位調製後的頻率分佈。第四節探討如何降低傳輸中的光子與原子之交互作用,模擬不同展頻頻率與原子吸收率之關係。


\section{基本原理介紹}
展頻技術 (spread spectrum technology) 是一種可將原訊號的頻譜打散分佈到比原始頻寬更寬的技術。我們的實驗選用的方法為直接序列展頻,能將窄頻雷射 (narrow-band laser) 與單光子的頻寬從約 10 MHz 展至 10 GHz,其作法為,以 PRBS(pseudo-random binary sequence)產生器產生高頻隨機訊號,並使用光電調製器 (EOM) 對入射光進行相位調製,此在時域上的操作,經傅立葉轉換後等效於增加其他頻率成分,以達到展寬頻率的效果。

\section{相位調製}

\subsection{數學形式}
此小節介紹相位調製的數學形式。設入射 EOM 的雷射波函數為 $E_{0}(t)$,調製函數 (modulation function) 為 $M(t)$,經調製後的波函數 $E_{m}(t)$ 可表示成:
\begin{equation}
    E_{m}(t)=E_{0}(t)e^{iM(t)}
\end{equation}
若對此式做傅立葉轉換,根據 convolution theorem,可得:
\begin{equation}
\label{eq:modulation_function}
    \mathscr{F}\{E_{0}(t)e^{iM(t)}\}=\tilde{E_{0}}(\omega)*\mathscr{F}\{{e^{iM(t)}}\}
\end{equation}
$\tilde{E_{0}}(\omega)$ 為入射光之頻譜,所以在頻譜數學分析上,我們可以把入射光頻譜與相位調製的部分分開處理,個別將兩項計算好後再做摺積即可得到調製後的頻譜。

\subsection{以單頻波之相位調製為例}
若入射光的頻譜為中心頻率在 $\nu_{0}$ 的勞倫茲分佈(lorenz distribution),調製函數為頻率 $\nu_{m}$ 的單頻波,意即輸入的電訊號強度隨時間的函數可表示為 $\phi_{0} sin(2\pi \nu_{m} t)$,則可將\cref{eq:modulation_function} 改寫為:
\begin{equation}
\label{eq:convolution_theorem}
    \mathscr{F}\{E_{0}(t)e^{i\phi_{0} sin(2\pi \nu_{m} t)}\}=\tilde{E_{0}}(\omega)*\mathscr{F}\{{e^{i\phi_{0} sin(2\pi \nu_{m} t)}}\}
\end{equation}
其中 $\tilde{E_{0}}(\omega)$ 為勞倫茲分佈,另一項傅立葉轉換的結果為第一類貝索函數(Bessel function of the first kind):
\begin{equation}
    \mathscr{F}\{{e^{i\phi_{0} sin(2\pi \nu_{m} t)}}\}=J_{n}(\phi_{0})
\end{equation}
或在時域上看,將調製項做傅立葉級數展開:
\begin{equation}
    e^{i\phi_{0} sin(2\pi \nu_{m} t)}=\sum_{n=-\infty}^{\infty}J_{n}(\phi_{0})e^{i 2 \pi n \nu_{m} t}
\end{equation}
可從上式看出,調製項的頻譜是由頻率為 $n \nu_{m}$ 的狄拉克函數(Dirac function) 組成,$n=0, \pm1, \pm2, ...$,強度分佈為 $J_{n}(\phi_{0})$,如\cref{fig:bessel_function},可看出在不同的調製電壓(也就是不同的$\phi_{0}$)時會有不一樣強度的頻率分布,以 $\phi_{0}=0.75\pi$ 為例,模擬對一道頻寬為 60 MHz 的雷射做 1 GHz 的正弦波相位調製,從\cref{eq:convolution_theorem} 可知,將入射光(\cref{fig:monocromatic_modulation} 黑線)與調製項的頻譜做摺積可得調製後的結果(\cref{fig:monocromatic_modulation} 紅線),每個頻率的間隔為 1 GHz。兩者比較可明顯看出,時域上相位調製能讓改變頻率的分佈。

\fig[0.75][fig:bessel_function][!htb]{bessel_function.png}[貝索函數]

\fig[0.75][fig:monocromatic_modulation][!htb]{monocromatic_modulation.png}[單頻波相位調製。黑線為調製前;紅線為調製後。][單頻波相位調製]


\section{展頻及壓縮}
\label{section:time_delay}
從上一節單頻波的例子可看出,相位調製可將原先頻率集中於 $\nu_{0}$ 的光,分散至 $\nu_{0}\pm\nu_{m}, \nu_{0}\pm2\nu_{m},\dots$。若調製函數改用時間寬度為 $\Delta T$ 的隨機方波 $PRBS(t)$ (如\cref{fig:PRBS_simulation}),則可將\cref{eq:modulation_function} 的右式寫成:

\begin{equation}
    \tilde{E_{0}}(\omega)*\mathscr{F}\{{e^{i PRBS(t)}}\}
\end{equation}
經計算,展寬後的頻譜如\cref{fig:spread_sprectrum_simulation} 藍線,其包絡線為 $sinc$ 的平方(黑色虛線),展開的寬度為 $\pm\frac{1}{\Delta T}$,在我們實驗中使用的隨機訊號的產生率為 10 Gb/s,單一位元的時間寬度為 100 ps,相當於能將頻譜從 10 MHz 展至 10 GHz 寬。

\fig[0.6][fig:PRBS_simulation][!htb]{prbs.png}[隨機訊號 $PRBS(t)$,為隨機產生之二位元訊號,每個位元的時間寬度為 $\Delta T$。實驗上會以隨機訊號產生器產生出振幅分別為 $\frac{1}{2}V_\pi$ 與 $-\frac{1}{2}V_\pi$ 之電訊號,能用來對光的相位做 $\pm \frac{1}{2}\pi$ 之調製] 

\fig[0.75][fig:spread_sprectrum_simulation][!htb]{spread_sinc.png}[展寬後頻譜模擬圖]

經展頻後的訊號,在傳輸的過程中可以降低環境的影響,避免光子被特定原子團吸收,但若想還原光子初始相位的資訊,則需要做一個反向的相位調製,讓光子再經過第二台 EOM,輸入的電訊號必須為與 $PRBS(t)$ 互補的 $\overline{PRBS}(t)$,這兩個訊號要滿足以下關係:

\begin{equation}
    \label{eq:prbs_condition}
    PRBS(t)+\overline{PRBS}(t)=0
\end{equation}
或

\begin{equation}
    e^{i PRBS(t)}\times e^{i \overline{PRBS}(t)}=1
\end{equation}

若光子在兩台 EOM 行經的時間間距為 $\Delta t_{p}$,兩個電訊號抵達的時間差為 $\Delta t_{RF}$,當 $\Delta t_{p}=\Delta t_{RF}$ 時,理論上可以對相位進行反向的調製,將展頻後的訊號壓縮,還原成原本的頻率分布,但若 $\Delta t_{p}>\Delta t_{RF}$ 或 $\Delta t_{p}<\Delta t_{RF}$,則無法完全還原頻譜,比較如\cref{fig:electric_mismatch},所以在實驗架設上,必須要能精確的控制電路與光路的長度,讓兩個電訊號匹配,才能達到最好的還原效果。

\todo[inline]{給更明顯的例子,附上模擬的參數}

\fig[0.7][fig:electric_mismatch][!htb]{phase_mismatch.png}[不同電訊號時間差之頻譜壓縮。黑線為 $\Delta t_{p}=\Delta t_{RF}$;紅線為 $\Delta t_{p} \neq \Delta t_{RF}$。][不同電訊號時間差之頻譜壓縮]

\section{$^{87}Rb$ 原子氣體吸收}

\subsection{展頻對吸收率的影響}
\label{section:simulation_absorption}
在光通訊中,以光作為資訊的載體,在空氣中傳輸的過程中光子會與原子產生交互作用,當光子的頻率接近原子的耀遷能階時有很大的機率會被吸收。以波長約為 795 nm 的窄頻雷射為例,將功率為 1 $\mu W$ 的光打入溫度約 70 度的 $^{87}Rb$ 原子氣體管,調整入射光頻率測量穿透率即可掃出 $^{87}Rb$ 的吸收譜,結果如\cref{fig:rb87_abs},從圖中可知,在頻率 377105000 MHz 與 377112000 MHz 的頻率位置分別約有 2GHz 與 1 GHz 寬的吸收區域,其吸收的中心頻率是被原子的能階給決定,可從飽和吸收光譜 (saturated absorption spectroscopy) 得知;吸收的寬度則是與原子蒸氣壓和溫度有關,不同的原子運動速度分佈會有不一樣的寬度,此為效應都卜勒增寬 (Doppler broadening)。

\fig[0.75][fig:rb87_abs][!htb]{absorption_spectrum.png}[$^{87}Rb$ 原子吸收譜,圖上的標示為躍遷能階]

為降低環境對光子的影響,我們可用上述之展頻技術,對光進行相位調製,將頻譜展寬,減少光對原子的吸收率。我們分別使用 1 Gb/s、5 Gb/s、10 Gb/s 與 20 Gb/s 的隨機訊號去模擬,在有展頻的狀態下,中心頻率與穿透率之關係,結果如\cref{fig:different_gbs_transmission},未經調製的光在 377105000 MHz 與 377112000 MHz 附近會被完全吸收,若將光的頻譜展寬則能顯著的降低吸收率,隨機訊號的頻率越高,原子對光子的影響越小。

\fig[0.75][fig:different_gbs_transmission][!htb]{different_gbs_spectroscopy.png}[展頻頻寬對吸收之影響。使用越快的隨機訊號對光進行相位調製,能降低光在原子躍遷能階附近的吸收率。][展頻頻寬對吸收之影響]

\subsection{原子吸收對頻譜還原之影響}
如前所述,對已調製過的光進行反向的調製,理論上可將頻譜壓窄,完美還原成調製前的頻率分佈。但若先讓已展頻的光通過原子團使其被部分吸收,再進行反向的調製,則還原回來的頻譜會與原始的有些差異,比較如\cref{fig:compress_comparison}。

\fig[0.75][fig:compress_comparison][!htb]{abs_compress_compare.png}[在沒放 $^{87}Rb$ 原子氣體管時,兩次的相位調製可將頻譜還原成初始的狀態,如圖黑線,但若在光路中加上 $^{87}Rb$ 原子氣體管,讓已展頻的光通過原子團,則會有部分的光被吸收,使光無法經由第二次的相位調製還原成最初的狀態,如圖紅線。][原子吸收對頻譜壓縮之影響]
\todo[inline]{多畫幾張不同吸收頻寬的結果並做比較。}

\end{document}