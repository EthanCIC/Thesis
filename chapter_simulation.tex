\documentclass[class=NCU_thesis, crop=false]{standalone}
\usepackage{mathrsfs,amsmath}

\begin{document}

\chapter{理論模擬}
\section{展頻及壓縮}
從上一章單頻波的例子可看出,相位調製可將原先頻率集中於 $\nu_{0}$ 的光,分散至 $\nu_{0}\pm\nu_{m}, \nu_{0}\pm2\nu_{m},\dots$。若調製函數改為時間寬度為 $\Delta T$ 的隨機方波 $PRBS(t)$ (如圖),則可將將 (\ref{eq:modulation_function}) 的右式寫成:
\begin{equation}
    \tilde{E_{0}}(\omega)*\mathscr{F}\{{e^{i PRBS(t)}}\}
\end{equation}
經計算,展寬後的頻譜如下:

\fig[0.5][fig:label][!htb]{temp.png}[展寬後頻譜模擬圖]

其包絡線接近 $sinc$ 的平方,展開的寬度為 $\pm\frac{1}{\Delta T}$,在我們實驗中使用的隨機訊號的產生率為 10 Gb/s,單一比特的時間寬度為 100 ps,相當於能將頻譜從數 MHz 展至 10 GHz 寬。

經展頻後的訊號,可以降低環境的影響,避免光子被特定原子團吸收,但若想還原光子初始相位的資訊,則需要一個反向的操作,讓光子再經過第二台相位調製器,輸入的電訊號為與 $PRBS(t)$ 互補的訊號 $\overline{PRBS}(t)$,這兩個訊號要滿足以下關係:
\begin{equation}
    PRBS(t)\times \overline{PRBS}(t)=1
\end{equation}
\todo[inline]{這樣寫有問題,重新想要如何以數學表達互補的訊號}
若光子在兩台相位調製器行經的時間間距為 $\Delta t_{p}$,兩個電訊號抵達的時間差為 $\Delta t_{RF}$,當 $\Delta t_{p}=\Delta t_{RF}$ 時,理論上可將展頻後的訊號壓縮回原本的樣子,但若 $\Delta t_{p}>\Delta t_{RF}$,則無法完全還原頻譜,如下圖:

\fig[0.5][fig:label][!htb]{temp.png}[$\Delta t_{p}>\Delta t_{RF}$ 時壓縮頻譜]

\section{銣原子氣體吸收}
從銣原子吸收譜可以看出,在其中兩個特定頻率上,各有約 1.5 GHz 的都卜勒吸收區,未經調製前的窄頻雷射進入原子氣體內會幾乎全部被吸收,但若將頻率展至 10 GHz 寬,則其中只有少部分會被吸收,這即是展頻的主要用途,可以降低光子被環境的影響,模擬圖如下:

\fig[0.5][fig:label][!htb]{temp.png}[展寬後頻譜模擬圖]

此時再讓光經過第二台相位調製器壓縮頻譜,結果比較圖如下:

\fig[0.5][fig:label][!htb]{temp.png}[展寬後頻譜模擬圖]

從圖上可以看出,光被部分吸收後,雖然還是能將頻譜還原成窄頻,但整體的功率會下降。

\end{document}