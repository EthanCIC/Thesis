\documentclass[class=NCU_thesis, crop=false]{standalone}
\usepackage{mathrsfs,amsmath}

\begin{document}

\chapter{理論模擬}
\section{展頻及壓縮}
從上一章單頻波的例子可看出,相位調製可將原先頻率集中於 $\nu_{0}$ 的光,分散至 $\nu_{0}\pm\nu_{m}, \nu_{0}\pm2\nu_{m},\dots$。若調製函數改用時間寬度為 $\Delta T$ 的隨機方波 $PRBS(t)$ (如\cref{fig:PRBS_simulation}),則可將\cref{eq:modulation_function} 的右式寫成:
\begin{equation}
    \tilde{E_{0}}(\omega)*\mathscr{F}\{{e^{i PRBS(t)}}\}
\end{equation}
經計算,展寬後的頻譜如\cref{fig:spread_sprectrum_simulation},其包絡線接近 $sinc$ 的平方,展開的寬度為 $\pm\frac{1}{\Delta T}$,在我們實驗中使用的隨機訊號的產生率為 10 Gb/s,單一位元的時間寬度為 100 ps,相當於能將頻譜從數 MHz 展至 10 GHz 寬。

\fig[0.5][fig:PRBS_simulation][!htb]{prbs.png}[隨機訊號 $PRBS(t)$]

\fig[0.5][fig:spread_sprectrum_simulation][!htb]{temp.png}[展寬後頻譜模擬圖]

經展頻後的訊號,在傳輸的過程中可以降低環境的影響,避免光子被特定原子團吸收,但若想還原光子初始相位的資訊,則需要做一個反向的相位調製,讓光子再經過第二台 EOM,輸入的電訊號必須為與 $PRBS(t)$ 互補的 $\overline{PRBS}(t)$,這兩個訊號要滿足以下關係:

\begin{equation}
    \label{eq:prbs_condition}
    PRBS(t)+\overline{PRBS}(t)=0
\end{equation}
或

\begin{equation}
    e^{i PRBS(t)}\times e^{i \overline{PRBS}(t)}=1
\end{equation}

若光子在兩台 EOM 行經的時間間距為 $\Delta t_{p}$,兩個電訊號抵達的時間差為 $\Delta t_{RF}$,當 $\Delta t_{p}=\Delta t_{RF}$ 時,理論上可以對相位進行反向的調製,將展頻後的訊號壓縮,還原成原本的頻率分布,但若 $\Delta t_{p}>\Delta t_{RF}$,則無法完全還原頻譜,比較如\cref{fig:electric_mismatch},所以在實驗架設上,必須要能精確的控制電路與光路的長度,讓兩個電訊號匹配,才能達到最好的還原效果。

\fig[0.5][fig:electric_mismatch][!htb]{temp.png}[$\Delta t_{p}>\Delta t_{RF}$ 時壓縮頻譜]

\section{$^{87}Rb$ 原子氣體吸收}

\subsection{展頻對吸收率的影響}
\label{section:simulation_absorption}
在光通訊中,以光作為資訊的載體,在空氣中傳輸的過程中光子會與原子產生交互作用,當光子的頻率接近原子的耀遷能階時有很大的機率會被吸收。以波長約為 795  nm 的窄頻雷射為例,將此道光打入溫度約 70 度的 $^{87}Rb$ 原子氣體管,調整入射光頻率測量穿透率即可掃出 $^{87}Rb$ 的吸收譜,結果如\cref{fig:rb87_abs},從圖中可知,在頻率 105 GHz 與 112 GHz 的頻率位置分別約有 2GHz 與 1 GHz 寬的吸收區域,其吸收的中心頻率是被原子的能階給決定,可從飽和吸收光譜 (saturated absorption spectroscopy) 得知;吸收的寬度則是與原子蒸氣壓和溫度有關,不同的原子運動速度分佈會有不一樣的寬度,此為效應都卜勒增寬 (Doppler broadening)。

\fig[0.75][fig:rb87_abs][!htb]{absorption_spectrum.png}[$^{87}Rb$ 原子吸收譜]

為降低環境對光子的影響,我們可用上述之展頻技術,對光進行相位調製,將頻譜展寬,減少光對原子的吸收率。我們分別使用 1 Gb/s、5 Gb/s、10 Gb/s 與 20 Gb/s 的隨機訊號去模擬,在有展頻的狀態下,中心頻率與穿透率之關係,結果如\cref{fig:different_gbs_transmission},未經調製的光在 105 GHz 與 112 GHz 附近會被完全吸收,若將光的頻譜展寬則能顯著的降低吸收率,隨機訊號的頻率越高,原子對光子的影響越小。

\fig[0.75][fig:different_gbs_transmission][!htb]{different_gbs_spectroscopy.png}[展頻頻寬對吸收之影響。使用越快的隨機訊號對光進行相位調製,能降低光在原子躍遷能階附近的吸收率。][展頻頻寬對吸收之影響]

\subsection{吸收對頻譜還原的影響}
如前所述,對已調製過的光進行反向的調製,理論上可將頻譜壓窄,完美還原成調製前的頻率分佈。但若先將已展頻的光通入原子團使其被部分吸收,再進行反向的調製,則還原回來的頻譜會與原先有些微的差異,比較如\cref{fig:compress_comparison}。

\fig[0.5][fig:compress_comparison][!htb]{temp.png}[展頻後吸收對壓縮之影響]

\end{document}