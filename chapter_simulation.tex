\documentclass[class=NCU_thesis, crop=false]{standalone}
\usepackage{mathrsfs,amsmath}

\begin{document}

\chapter{理論模擬}
\section{展頻及壓縮}
從上一章單頻波的例子可看出,相位調製可將原先頻率集中於 $\nu_{0}$ 的光,分散至 $\nu_{0}\pm\nu_{m}, \nu_{0}\pm2\nu_{m},\dots$。若調製函數改用時間寬度為 $\Delta T$ 的隨機方波 $PRBS(t)$ (如圖),則可將 (\ref{eq:modulation_function}) 的右式寫成:
\begin{equation}
    \tilde{E_{0}}(\omega)*\mathscr{F}\{{e^{i PRBS(t)}}\}
\end{equation}
經計算,展寬後的頻譜如圖:

\fig[0.5][fig:label][!htb]{temp.png}[隨機訊號 $PRBS(t)$]

\fig[0.5][fig:label][!htb]{temp.png}[展寬後頻譜模擬圖]

其包絡線接近 $sinc$ 的平方,展開的寬度為 $\pm\frac{1}{\Delta T}$,在我們實驗中使用的隨機訊號的產生率為 10 Gb/s,單一位元的時間寬度為 100 ps,相當於能將頻譜從數 MHz 展至 10 GHz 寬。

經展頻後的訊號,可以降低環境的影響,避免光子被特定原子團吸收,但若想還原光子初始相位的資訊,則需要做一個反向的相位調製,讓光子再經過第二台 EOM,輸入的電訊號必須為與 $PRBS(t)$ 互補的 $\overline{PRBS}(t)$,這兩個訊號要滿足以下關係:
\begin{equation}
    PRBS(t)+\overline{PRBS}(t)=0
\end{equation}
或

\begin{equation}
    e^{i PRBS(t)}\times e^{i \overline{PRBS}(t)}=1
\end{equation}

若光子在兩台 EOM 行經的時間間距為 $\Delta t_{p}$,兩個電訊號抵達的時間差為 $\Delta t_{RF}$,當 $\Delta t_{p}=\Delta t_{RF}$ 時,理論上可以對相位進行反向的調製,將展頻後的訊號壓縮回原本的樣子,但若 $\Delta t_{p}>\Delta t_{RF}$,則無法完全還原頻譜,如下圖:

\fig[0.5][fig:label][!htb]{temp.png}[$\Delta t_{p}>\Delta t_{RF}$ 時壓縮頻譜]

\section{銣原子氣體吸收}
從銣原子吸收譜可以看出,在其中兩個特定頻率上,各有約 1.5 GHz 的都卜勒吸收區,未經調製前的窄頻雷射進入原子氣體內會幾乎全部被吸收,但若將頻率展至 10 GHz 寬,則其中只有少部分會被吸收,這即是展頻的主要用途,可以降低光子被環境的影響,模擬如圖:

\fig[0.5][fig:label][!htb]{temp.png}[展寬後頻譜模擬圖]

此時再讓光經過第二台 EOM 將頻譜壓縮,結果比較圖,可以看出,儘管光有被部分吸收,還是能透過反向的調製將已展寬的頻譜變窄。

\fig[0.5][fig:label][!htb]{temp.png}[壓縮後頻譜模擬圖]

\end{document}