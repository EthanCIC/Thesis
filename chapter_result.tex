\documentclass[class=NCU_thesis, crop=false]{standalone}
\begin{document}
\chapter{實驗架設與結果討論}


\section{光源製備}

\subsection{雷射光}
雷射光源為 Toptica 的半導體雷射,可產生波長 795 nm 的窄頻雷射

\subsection{單光子}
雙光子的產生機制為 SPDC,入射一道波長 397.5 奈米的藍光雷射進入 PPKTP 晶體,產生 Type-II 的時間 - 能量糾纏光子對 (time-energy entangled biphoton),波長為 795 奈米。
實驗上會將產生出來的雙光子對經過 PBS,將訊號分為 signal 和 idler,以 idler 做為觸發訊號,讓 signal 經過 $^{87}Rb$ 原子氣體管與 EOM,讓光子被吸收或對其進行相位的調製,並做 $G^{2}(\tau)$ 的測量,$G^{2}(\tau)$ 的定義如下。

\begin{equation}
    G^{2}(\tau)=\frac{4\Gamma_{s}\Gamma_{i}}{\Gamma_{s}+\Gamma_{i}}\left\{\begin{matrix}
        e^{\Gamma_{s}\tau} & ,\tau<0\\
        e^{-\Gamma_{i}\tau} & ,\tau>0
        \end{matrix}\right.
\end{equation}

此為二階強度關聯函數 (second-order intenstiy correlation function),$\tau$ 為兩顆單光子抵達探測器的時間差。實際測量結果如\cref{fig:single_photon_g2},此光子之時間波包寬度約為 100 ns,頻寬為 4.5 MHz。

\fig[0.5][fig:single_photon_g2][!htb]{temp.png}[雷射頻譜量測光路圖]

\section{雷射頻譜量測}

實驗光路架設如圖,我們將窄頻雷射通過兩台 EOM 對其進行相位調製,第一台為展頻用,第二台用來做反向的調製還原頻譜,再以 Fabrty-Perot 干涉儀去測量頻譜。

\fig[0.5][fig:label][!htb]{temp.png}[雷射頻譜量測光路圖]

在兩台 EOM 都關閉的情況下,可以測到波長 795 奈米窄頻雷射的頻譜,結果如圖,以此 Fabry-Perot 的解析度掃出的雷射頻寬約為 30 MHz。

\fig[0.5][fig:label][!htb]{temp.png}[窄頻雷射頻譜]

若只開啟第一台 EOM,在 10 Gb/s 隨機訊號的調製下可將窄頻雷射光的頻譜展至 10 GHz 寬,但由於我們的使用的 Fabry-Perot FSR 僅 10 GHz,無法涵蓋完整的頻率區間,會使測量的結果失真,要想掃出完整展開的頻譜需使用 FSR 20 GHz 以上的干涉儀,所以下面會先以 2 Gb/s 的訊號來測試展頻的結果是否符合理論模擬。

\subsection{2 Gb/s 隨機訊號之相位調製}
先以 2 Gb/s 隨機訊號進行相位調製,只開啟第一台能將頻譜展至 $\pm$5 GHz 寬,如下圖。

\fig[0.5][fig:label][!htb]{temp.png}[5 Gb/s 訊號之展頻頻譜]

頻譜的形狀大致上與理論相符,但在 $\pm$2 GHz 的位置有一個突起的訊號,這是由於隨機訊號的上升與下降時間不夠快所致,若在數值模擬中把隨機訊號加上約 30 ps 的上升與下降時間(如圖),則會出現類似的結果,如圖:

\begin{figure}[!hbt]
    %\captionsetup[subfigure]{labelformat=empty} % 完全隱藏圖號
    \centering
    \subcaptionbox
        {caption\_1
        \label{fig:subfig_fig1}}
        {\includegraphics[width=0.3\linewidth]{temp.png}}
    ~~~~
    \subcaptionbox
        {caption\_2
        \label{fig:subfig_fig2}}
        {\includegraphics[width=0.3\linewidth]{temp.png}}
\end{figure}

% \fig[0.5][fig:label][!htb]{temp.png}[修正前後之隨機訊號]
% \fig[0.5][fig:label][!htb]{temp.png}[修正後展頻頻譜]

此外,還可看出該頻譜的包絡線有週期振盪的訊號,原因為我們使用的隨機訊號實際上是個重複出現的週期訊號,每個週期有 $2^{31}-1$ 個位元,若把單位週期的位元數調為 $2^{15}-1$ 做可看到週期更小的震盪週期,如圖:

\begin{figure}[!hbt]
    %\captionsetup[subfigure]{labelformat=empty} % 完全隱藏圖號
    \centering
    \subcaptionbox
        {caption\_1
        \label{fig:subfig_fig1}}
        {\includegraphics[width=0.3\linewidth]{temp.png}}
    ~~~~
    \subcaptionbox
        {caption\_2
        \label{fig:subfig_fig2}}
        {\includegraphics[width=0.3\linewidth]{temp.png}}
\end{figure}

從測量的頻譜可以看出,展寬的頻率與理論計算的結果一致,所以我們認為 10 Gb/s 的隨機訊號能將訊號展至 $\pm$10 GHz 寬。

\subsection{10 Gb/s 隨機訊號之相位調製}

上一小節我們先以 2 Gb/s 的訊號做展頻的測試,是由於我們使用的 Fabry-Perot 干涉儀 FSR 不夠大,無法涵蓋以 10 Gb/s 訊號調製的展頻頻譜。至於壓縮頻譜的部分,能將頻寬還原成約 10 MHz,所以可使用 10 Gb/s 的訊號進行調製與量測。

當兩台 EOM 同時開啟時,理論上要能將展寬的頻譜還原成調製前的狀態,但從(圖)的實驗結果可以看出,壓縮回來的頻譜與調製前相比,中心頻率的強度僅為本來的 70\%,若只計算中心頻率附近 1 GHz 的頻率區間,與調製前的頻譜相比光強僅約 80\%,剩下 20\% 的能量還分散在其他頻率沒被還原。造成頻譜壓縮效果不佳的可能原因為,兩個隨機訊號的形狀不同,上下也不夠對稱,導致無法將相位做反向的調製,使訊號完美還原成最初的狀態。

\fig[0.5][fig:label][!htb]{temp.png}[10 Gb/s 訊號壓縮後頻譜]

\section{$^{87}Rb$ 原子吸收譜}

為了確定相位調製對於銣原子吸收的影響,我們在兩台 EOM 的後面放上一個 $^{87}Rb$ 原子氣體管,並以光二極體 (photodiode) 收光,測量透射的強度。只有在入射頻率與銣原子躍遷能階共振時光子才會被吸收,使透射率降低,所以若連續調整入射光的頻率,則能掃出整個吸收譜,如\cref{fig:abs_spec} 黑線。

\fig[1][fig:abs_spec][!htb]{absorption_spectrum.png}[調製後的銣原子吸收譜]

接著打開第一台 EOM,將頻寬從 30 MHz 展至 10 GHz,結果如\cref{fig:abs_spec} 紫線,可見頻譜展寬之後,光能大部分透射銣原子氣體不被吸收,就像隱形了一樣,能降低光子受環境的影響。
若同時開啟兩台 EOM 將頻譜壓縮,則能再次看到光被吸收,如\cref{fig:abs_spec} 紅線,但吸收率卻明顯降低,原因如上一小節所述,可能為隨機訊號品質不所致,影響頻譜壓縮的效果,有部分的能量還分散在各個頻率上沒能被還原,那些能量不在銣原子的共振頻率上,所以能夠穿透氣體管,使穿透率上升。
\todo[inline]{重畫圖,把藍線去掉}

\section{單光子相位調製對原子吸收之影響}
從前一小節的實驗結果能得知,$^{87}Rb$ 的躍遷頻率約在 105 GHz 與 112 GHz 附近,這時我們將光源從窄頻雷射換成單光子,並透過改變入射光的頻率與晶體溫度,將單光子的頻率調至 112 GHz,使其能被原子吸收,再以\cref{fig:single_photon_no_etalon}的光路架設,對光子進行相位調製與測量。

\fig[0.5][fig:single_photon_no_etalon][!htb]{temp.png}[單光子量測光路圖]

當兩台 EOM 皆關閉時,頻寬約為 4.5 MHz 的單光子會幾乎完全被原子吸收,光無法透射氣體管,但若對其進行 $G^{2}(\tau)$ 測量,卻會測到訊號,如\cref{fig:remaining_mod_g2},這是由於我們晶體產生的單光子源非單模 (single-mode),其中還存在符合別組相位匹配條件 (phase-matching condition) 產生的光,若要去除那些光子對實驗的影響,在此小節的數據處理上,我們直接將其當作雜訊扣除;下一小節的實驗中,我們會外加一個 Etalon 濾波器,只讓 112 GHz 附近的光通過。

\fig[0.5][fig:remaining_mod_g2][!htb]{temp.png}[單光子通過 $^{87}Rb$ 氣體管之 $G^{2}(\tau)$ 量測]

若開啟第一台 EOM,使用 10 Gb/s 的隨機訊號對單光子進行相位調製,可以讓單光子的頻寬從 4.5 MHz 展至 10 GHz,使大部分的光可以透射 $^{87}Rb$ 氣體不被吸收,扣除雜訊後的 $G^2(\tau)$ 的測量如\cref{fig:spread_absorption_g2},透射率為 76\%。另外,此時若將 $^{87}Rb$ 氣體管移除,直接測量展頻後的訊號,能發現單位時間測量到的光子數與調製前相差不多,印證了本章第一小節的結論——相位調製不影響光強。

\fig[0.5][fig:spread_absorption_g2][!htb]{temp.png}[展頻後單光子被部分吸收後之 $G^{2}(\tau)$ 量測]

\section{單光子頻譜壓縮}

從上一小節的結果可知,使用展頻技術可以有效的降低環境對光子的影響,但若考量到接收訊息端可能會需要光子原始的相位資訊,或者要讓光子與 $^{87}Rb$ 原子進行交互作用,我們必須要開啟第二台 EOM 進行反向的調製,盡量使光子還原到原先的狀態,若以\cref{fig:single_photon_no_etalon}的光路架設,將第二台 EOM 開啟,由於相位調製不影響光強,無從得知頻寬是否有被還原,因此要將光路架設改為\cref{fig:single_photon_with_etalon},在單光子探測器前加上 Etalon 濾波器,限制只讓頻寬 60 MHz 內的光通過,如此一來,只要能測到訊號就代表部分光子的頻寬有被壓窄至 60 MHz 內,另一方面,這也可以將上一小節及提的雜訊去除。

\fig[0.5][fig:single_photon_with_etalon][!htb]{temp.png}[加上濾波器之單光子量測光路圖]

以\cref{fig:single_photon_with_etalon}的光路架設,只開啟第一台 EOM 時,被展頻的單光子能大部分透射原子團,但由於 Etalon 的過濾,頻寬 10 GHz 的光子幾乎無法抵達探測器,因而測不到明顯的訊號,結果如\cref{fig:spread_single_photon_with_etalon}。若將第二台 EOM 也開啟,將已展頻的單光子頻譜壓縮,則能再次測到訊號,如\cref{fig:compress_single_photon_with_etalon}

\begin{figure}[!hbt]
    %\captionsetup[subfigure]{labelformat=empty} % 完全隱藏圖號
    \centering
    \subcaptionbox
        {caption\_1
        \label{fig:spread_single_photon_with_etalon}}
        {\includegraphics[width=0.3\linewidth]{temp.png}}
    ~~~~
    \subcaptionbox
        {caption\_2
        \label{fig:compress_single_photon_with_etalon『}}
        {\includegraphics[width=0.3\linewidth]{temp.png}}
\end{figure}

為了知道原子團的吸收對於單光子頻譜的壓縮有何影響,我們在兩台 EOM 同時開啟時將氣體管移除,測得的結果比較如\cref{fig:no_cell_compress},有 25\% 的單光子可以在被部分吸收後,重新將頻譜壓縮回 60 MHz 內。

\fig[0.5][fig:no_cell_compress][!htb]{temp.png}[原子吸收對單光子壓縮品質比較圖]


\end{document}