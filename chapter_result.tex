\documentclass[class=NCU_thesis, crop=false]{standalone}
\begin{document}

\chapter{實驗架設與結果討論}

\section{相位調製對光強的影響}
我們先以兩種方式去確認相位調製對光強的影響,第一種是用功率計 (power meter) ,讓光一次通過兩台 EOM,去分別測量兩台同時開啟與關閉時的光強,再將兩個數值相除得到變化率;另一種方法是透過 Fabry-Perot 測量頻譜,比較相位調製前後的總面積大小,測量結果如下。

\fig[0.75][fig:label][!htb]{eom-transmission.png}[EOM 開啟前後之功率變化率]

從隨機訊號 1 Gb/s 到 10 Gb/s 的調製,看不出有特定的變化趨勢,可見展頻與壓縮只會影響頻率的分佈,不會明顯改變光強度。

\section{古典光量測}
古典光源為 Toptica 的半導體雷射,可產生波長 795 nm 的窄頻雷射,這部分的量測主要可以分成兩個部分,第一個部分是要測量雷射的頻譜,看相位調製如何影響頻率的分佈;另一部分是讓雷射通過銣原子氣體管,調整入射雷射的頻率測量 $Rb^{87}$ 的吸收譜。

\subsection{雷射頻譜量測}

實驗光路架設如圖,我們將窄頻雷射通過兩台 EOM 對其進行相位調製,第一台為展頻用,第二台用來做反向的調製來還原頻譜,再以 Fabrty-Perot 干涉儀去掃頻。

\fig[0.5][fig:label][!htb]{temp.png}[雷射頻譜量測光路圖]

在兩台 EOM 都關閉的情況下,可以測到波長 795 奈米窄頻雷射的頻譜,結果如圖,以此 Fabry-Perot 的解析度掃出的雷射頻寬約為 30 MHz。

\fig[0.5][fig:label][!htb]{temp.png}[窄頻雷射頻譜]

若只開啟第一台 EOM,在 10 Gb/s 隨機訊號的調製下可將窄頻雷射光的頻譜展至 10 GHz 寬,但由於我們的使用的 Fabry-Perot FSR 僅 10 GHz,無法涵蓋完整的頻率區間,會使測量的結果失真,要想掃出完整展開的頻譜需使用 FSR 20 GHz 以上的干涉儀,所以下面會先以 2 Gb/s 的訊號來測試展頻的結果是否符合理論模擬。

\subsubsection{2 Gb/s 隨機訊號之相位調製}
先以 2 Gb/s 隨機訊號進行相位調製,只開啟第一台能將頻譜展至 $\pm$5 GHz 寬,如下圖。

\fig[0.5][fig:label][!htb]{temp.png}[5 Gb/s 訊號之展頻頻譜]

頻譜的形狀大致上與理論相符,但在 $\pm$2 GHz 的位置有一個突起的訊號,這是由於隨機訊號的上升與下降時間不夠快所致,若在數值模擬中把隨機訊號加上約 30 ps 的上升與下降時間(如圖),則會出現類似的結果,如圖:

\begin{figure}[!hbt]
    %\captionsetup[subfigure]{labelformat=empty} % 完全隱藏圖號
    \centering
    \subcaptionbox
        {caption\_1
        \label{fig:subfig_fig1}}
        {\includegraphics[width=0.3\linewidth]{temp.png}}
    ~~~~
    \subcaptionbox
        {caption\_2
        \label{fig:subfig_fig2}}
        {\includegraphics[width=0.3\linewidth]{temp.png}}
\end{figure}

% \fig[0.5][fig:label][!htb]{temp.png}[修正前後之隨機訊號]
% \fig[0.5][fig:label][!htb]{temp.png}[修正後展頻頻譜]

此外,還可看出該頻譜的包絡線有週期振盪的訊號,原因為我們使用的隨機訊號實際上是個重複出現的週期訊號,每個週期有 $2^{31}-1$ 個位元,若把單位週期的位元數調為 $2^{15}-1$ 做可看到週期更小的震盪週期,如圖:

\begin{figure}[!hbt]
    %\captionsetup[subfigure]{labelformat=empty} % 完全隱藏圖號
    \centering
    \subcaptionbox
        {caption\_1
        \label{fig:subfig_fig1}}
        {\includegraphics[width=0.3\linewidth]{temp.png}}
    ~~~~
    \subcaptionbox
        {caption\_2
        \label{fig:subfig_fig2}}
        {\includegraphics[width=0.3\linewidth]{temp.png}}
\end{figure}

從測量的頻譜可以看出,展寬的頻率與理論計算的結果一致,所以我們認為 10 Gb/s 的隨機訊號能將訊號展至 $\pm$10 GHz 寬。

\subsubsection{10 Gb/s 隨機訊號之相位調製}

上一小節我們先以 2 Gb/s 的訊號做展頻的測試,是因為我們使用的 Fabry-Perot 干涉儀 FSR 不夠大,無法涵蓋以 10 Gb/s 訊號調製的展頻頻譜,所以先用 2 Gb/s 做確認。至於壓縮頻譜的部分,能將頻寬壓回數十 MHz,所以可使用 10 Gb/s 的訊號進行調製與量測。

當兩台 EOM 同時開啟時,理論上要能將展寬的頻譜還原成調製前的狀態,但從(圖)的實驗結果可以看出,壓縮回來的頻譜與調製前相比,中心頻率的強度僅為本來的 70\%,若只計算中心頻率附近 1 GHz 的頻率區間,與調製前的頻譜相比光強僅約 80\%,剩下 20\% 的能量還分散在其他頻率沒被還原。造成頻譜壓縮效果不佳的可能原因為,兩個隨機訊號的形狀不同,上下也不夠對稱,導致無法將相位做反向的調製,使訊號完美還原成最初的狀態。

\fig[0.5][fig:label][!htb]{temp.png}[10 Gb/s 訊號壓縮後頻譜]

\subsection{銣原子吸收譜}
為了確定相位調製對於銣原子吸收的影響,我們在兩台 EOM 的後面放上一個如原子氣體管,並以光二極體 (photodiode) 收光,測量透射的強度。只有在入射頻率與銣原子躍遷能階共振時光子才會被吸收,使透射率降低,所以若連續調整入射光的頻率,則能掃出整個吸收譜,如\cref{fig:abs_spec} 黑線。

\fig[1][fig:abs_spec][!htb]{absorption_spectrum.png}[調製後的銣原子吸收譜]

接著打開第一台 EOM,將頻寬從 30 MHz 展至 10 GHz,結果如\cref{fig:abs_spec} 紫線,可見頻譜展寬之後,光能大部分透射銣原子氣體不被吸收,就像隱形了一樣,能降低光子受環境的影響。
若同時開啟兩台 EOM 將頻譜壓縮,則能再次看到光被吸收,如\cref{fig:abs_spec} 紅線,但吸收率卻明顯降低,原因如上一小節所述,可能為隨機訊號品質不所致,影響頻譜壓縮的效果,有部分的能量還分散在各個頻率上沒能被還原,那些能量不在銣原子的共振頻率上,所以能夠穿透氣體管,使穿透率上升。
\todo[inline]{重畫圖,把藍線去掉}

\section{單光子量測}

\section{單光子光源製備}
雙光子的產生機制為 SPDC,入射一道波長 397.5 奈米的藍光雷射進入 PPKTP 晶體,產生 Type-II 的時間 - 能量糾纏光子對 (time-energy entangled biphoton),波長為 795 奈米。
實驗上會讓雙光子對經過 PBS 分光,並做 $G^{2}(\tau)$ 的測量,$G^{2}(\tau)$ 的定義如下。

\begin{equation}
    G^{2}(\tau)=\frac{4\Gamma_{s}\Gamma_{i}}{\Gamma_{s}+\Gamma_{i}}\left\{\begin{matrix}
        e^{\Gamma_{s}\tau} & ,\tau<0\\
        e^{-\Gamma_{i}\tau} & ,\tau>0
        \end{matrix}\right.
\end{equation}

此為二階強度關聯函數 (second-order intenstiy correlation function),$\tau$ 為兩顆單光子抵達探測器的時間差。

若調整入射光的頻率與 PPKTP 晶體的溫度,則可改變單光子的頻率。

雙光子在產生出來後會先進 PBS 將訊號分為 signal 和 idler,以 idler 做為觸發訊號,讓 signal 經過 EOM 與銣原子氣體管,進行相位的調製與吸收。

這部分的實驗,我們將單光子 (signal) 的頻率調至銣原子的躍遷頻率,使光子能被吸收,再藉由相位調製展寬光子的頻譜,降低原子對光子的影響,以達到隱形斗篷的效果。但由於單光子能量太弱,無法以 Fabry-Perot 干涉儀掃頻,所以為了要確定兩台 EOM 同時開啟時能否將頻譜還原,我們改用頻寬 60 MHz 的 Etalon 濾波器,限制光子能通過的頻寬,架設如圖。

\fig[0.5][fig:label][!htb]{temp.png}[單光子量測光路圖]

\subsection{展頻與壓縮}
以(光路圖)的架設,先不要放 $^{87}Rb$ 原子氣體管,讓單光子直接通過 60 MHz 寬的 Etalon 濾波器。若兩台 EOM 都沒開,窄頻的單光子能完全通濾波器,$G^{2}(\tau)$ 的量測結果如圖。此時若開啟第一台 EOM,將頻譜展至 10 GHz 寬,則單光子僅有極低的機率能通過 Etalon,如圖,幾乎測不到單光子的訊號。若同時開啟兩台 EOM,互補的隨機訊號能互相抵銷相位的變化,使頻譜還原至窄頻,如此就能使光子再次通過 Etalon,如圖,與未調製前的結果相比,強度低了些,這是由於壓縮效果不夠好,導致部分的光子沒能通過 Etalon 所致。

\fig[0.5][fig:label][!htb]{temp.png}[nocell 調製比較圖]

\subsection{$^{87}Rb$吸收}
同上一小節的光路架設,但把 $^{87}Rb$ 原子氣體管放回光路上,如圖。兩台相位調製器都不開的話,單光子幾乎全部被吸收,如圖。若開啟第一台 EOM 將單光子頻譜展寬,雖然光子能幾乎不被吸收,但由於 Etalon 的過濾,探測器仍測不太到光子,如圖。若將第二台相位調製器也開啟,則能把單光子的頻譜壓縮,再通過 Etalon,如圖。

\fig[0.5][fig:label][!htb]{temp.png}[heatcell 調製比較圖]

單獨將圖與圖拿出來比較 (圖),同樣是測量展頻再壓縮回來的光,黑線為沒放銣原子氣體館,紅線則有,從結果可看出,本來該被完全吸收的單光子,可透過相位的調製大幅降低銣原子氣體的影響,讓部分的光可以穿透,並在頻譜還原後被探測,達到隱形斗篷的效果。

\fig[0.5][fig:label][!htb]{temp.png}[調製後光子之銣原子吸收比較圖]

\end{document}