\documentclass[class=NCU_thesis, crop=false]{standalone}
\begin{document}

\chapter{實驗結果與討論}

\section{相位調製對光強的影響}
由此可知,展頻與壓縮只會影響頻率的分佈,不會明顯改變光強度。

\section{古典光量測}
\subsection{展頻與壓縮}
以(光路圖)的架設,若兩台相位調製器都不開的話,可以測到入射古典光的頻譜,如圖。若將第一台打開,則可以將頻譜展寬,看不到如(理論圖)的 sinc 平方是因為 Fabry-Perot 的 FSR 僅 10 GHz,想要掃出完整展開的頻譜需 FSR 20 GHz 以上才能看到。
兩台相位調製器同時開啟的話則可以將頻譜先展寬再壓縮回窄頻,但從實驗結果可看出,壓縮回來的訊號在 500 MHz 的頻率區間內,強度僅有約 70\%,原因可能為兩個隨機訊號的品質不同,也不夠對稱,導致無法將相位做反向的操作,還原成原本訊號的樣子。

\subsection{銣原子吸收譜}
為了確定相位調製對於銣原子吸收的影響,我們調整入射光的頻率,掃出整個吸收譜,如下圖黑線。
接著將第一台相位調製器打開,將頻寬從 30MHz 調至 10GHz,結果如上圖藍線,可見頻譜展寬之後,光能大部分透射銣原子氣體不被吸收。
若同時將兩台開啟,則能再次看到光被吸收,但吸收率卻明顯降低,原因如上一小節所述,可能為訊號不夠好影響壓縮品質所致。

\section{單光子量測}
\subsection{展頻與壓縮}
以(光路圖)的架設,先不要放 $^{87}Rb$ 原子氣體管,讓單光子直接通過 60 MHz 寬的 Etalon 濾波器。若兩台相位調製器都沒開的話,窄頻的單光子能完全通過,$G^{2}(\tau)$ 的量測結果如圖。此時若開啟第一台相位調製器,將頻譜展至 10 GHz 寬,則量子光僅有極低的機率能通過 Etalon,如圖。若將兩台相位調製器都開啟,互補的隨機訊號能互相抵銷相位的變化,使頻譜還原至窄頻,如此就能再次通過 Etalon,如圖。

\subsection{$^{87}Rb$吸收}
同上一小節的光路架設,但把 $^{87}Rb$ 原子氣體管放回光路上。兩台相位調製器都不開的話,單光子幾乎全部被吸收,如圖。若開啟第一台相位調製器將單光子頻譜展寬,雖然光子能幾乎不被吸收,但由於 Etalon 的過濾,探測器仍測不太到光子。若將第二台相位調製器也開啟,則能把單光子的頻譜壓縮,再通過 Etalon,如圖。

單獨將圖與圖拿出來比較如下,同樣是測量展頻再壓縮回來的光,有測量

\centering 

\end{document}