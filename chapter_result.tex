\documentclass[class=NCU_thesis, crop=false]{standalone}
\begin{document}

\chapter{實驗結果與討論}

\section{相位調製對光強的影響}
我們先以兩種方式去確認相位調製對光強的影響,第一種是用功率計 (power meter) ,去分別測量兩台 EOM 都開啟與關閉時的光強,再將兩個數值相除得到變化率;另一種方法是透過 Fabry-Perot 測量頻譜,比較相位調製前後的總面積大小,測量結果如下。

\fig[0.75][fig:label][!htb]{eom-transmission.png}[EOM 開啟前後之功率變化率]

從隨機訊號 1 Gb/s 到 10 Gb/s 的調製,看不出有特定的變化趨勢,可見展頻與壓縮只會影響頻率的分佈,不會明顯改變光強度。

\section{古典光量測}
\subsection{展頻與壓縮}
此實驗部分的實驗使用(光路圖)的架設,若只開啟第一台 EOM,在 10 Gb/s 隨機訊號的調製下可將窄頻雷射光的頻譜展至 10 GHz 寬,但由於我們的使用的 Fabry-Perot FSR 僅 10 GHz,無法涵蓋完整的頻率區間,要想掃出完整展開的頻譜需使用 FSR 20 GHz 以上的才能看到,所以下面會先用 2 Gb/s 的訊號來測試,展頻的結果是否符合理論模擬。

\subsubsection{入射光之頻譜}
在兩台 EOM 都關閉的情況下,可以測到波長 795 nm 窄頻雷射的頻譜,結果如下圖,以此 Fabry-Perot 的解析度掃出的雷射頻寬約為 30 MHz。

\fig[0.5][fig:label][!htb]{temp.png}[窄頻雷射頻譜]

\subsubsection{5 Gb/s 隨機訊號之相位調製}
先以 5 Gb/s 隨機訊號進行相位調製,只開啟第一台能將頻譜展至 $\pm$5GHz,如下圖。

\fig[0.5][fig:label][!htb]{temp.png}[5 Gb/s 訊號之展頻頻譜]

頻譜的形狀大致上與理論相符,但在 $\pm$5GHz 的位置有一個突起的訊號,這是由於隨機訊號的上升與下降時間不夠快所致,若在數值模擬中把隨機訊號加上約 30 ps 的上升與下降時間(如圖),則會出現類似的結果,如圖:

\begin{figure}[!hbt]
    %\captionsetup[subfigure]{labelformat=empty} % 完全隱藏圖號
    \centering
    \subcaptionbox
        {caption\_1
        \label{fig:subfig_fig1}}
        {\includegraphics[width=0.3\linewidth]{temp.png}}
    ~~~~
    \subcaptionbox
        {caption\_2
        \label{fig:subfig_fig2}}
        {\includegraphics[width=0.3\linewidth]{temp.png}}
\end{figure}

% \fig[0.5][fig:label][!htb]{temp.png}[修正前後之隨機訊號]
% \fig[0.5][fig:label][!htb]{temp.png}[修正後展頻頻譜]

此外,還可看出該頻譜的包絡線有週期振盪的訊號,其原因為我們使用的隨機訊號實際上是個重複出現的週期訊號,每個週期有 $2^{31}-1$ 個位元,若把單位週期的位元數調為 $2^{15}-1$ 做可看到週期更小的震盪週期,如圖:

\begin{figure}[!hbt]
    %\captionsetup[subfigure]{labelformat=empty} % 完全隱藏圖號
    \centering
    \subcaptionbox
        {caption\_1
        \label{fig:subfig_fig1}}
        {\includegraphics[width=0.3\linewidth]{temp.png}}
    ~~~~
    \subcaptionbox
        {caption\_2
        \label{fig:subfig_fig2}}
        {\includegraphics[width=0.3\linewidth]{temp.png}}
\end{figure}

當兩台 EOM 同時開啟時,理論上要能將展寬的頻譜還原成調製前的狀態,但從(圖)的實驗結果可以看出,壓縮回來的頻寬較
寬,若只計算中心頻率附近 500 MHz 的頻率區間,與調製前的相比,光強僅約 70\%,造成壓縮效果不佳的可能原因為兩個隨機訊號的形狀不同,上下也不夠對稱,導致無法將相位做反向的調製,使訊號完美還原成最初的狀態。

\paragraph{10 Gb/s 隨機訊號之相位調製}

從 5 Gb/s 的展頻可看出,展寬的頻率區間如理論計算的結果
頻譜還原的訊號,與調製前的頻譜比較如圖,若將雷射調至會被銣原子吸收的頻率,


\subsection{銣原子吸收譜}
為了確定相位調製對於銣原子吸收的影響,我們調整入射光的頻率,掃出整個吸收譜,如下圖黑線。
接著打開第一台 EOM,將頻寬從 30 MHz 展至 10 GHz,結果如上圖藍線,可見頻譜展寬之後,光能大部分透射銣原子氣體不被吸收。
若同時將兩台 EOM 開啟,則能再次看到光被吸收,但吸收率卻明顯降低,原因如上一小節所述,可能為訊號不夠好影響壓縮品質所致。

\section{單光子量測}
\subsection{展頻與壓縮}
以(光路圖)的架設,先不要放 $^{87}Rb$ 原子氣體管,讓單光子直接通過 60 MHz 寬的 Etalon 濾波器。若兩台 EOM 都沒開的話,窄頻的單光子能完全通濾波器,$G^{2}(\tau)$ 的量測結果如圖。此時若開啟第一台 EOM,將頻譜展至 10 GHz 寬,則量子光僅有極低的機率能通過 Etalon,如圖。若將兩台相位調製器都開啟,互補的隨機訊號能互相抵銷相位的變化,使頻譜還原至窄頻,如此就能再次通過 Etalon,如圖。

\subsection{$^{87}Rb$吸收}
同上一小節的光路架設,但把 $^{87}Rb$ 原子氣體管放回光路上。兩台相位調製器都不開的話,單光子幾乎全部被吸收,如圖。若開啟第一台相位調製器將單光子頻譜展寬,雖然光子能幾乎不被吸收,但由於 Etalon 的過濾,探測器仍測不太到光子。若將第二台相位調製器也開啟,則能把單光子的頻譜壓縮,再通過 Etalon,如圖。

單獨將圖與圖拿出來比較如下,同樣是測量展頻再壓縮回來的光,有測量

\centering 

\end{document}