\documentclass[class=NCU_thesis, crop=false]{standalone}
\begin{document}

\chapter{摘要}

在古典通訊中,展頻技術可以有效增加資訊傳輸過程中的隱匿性與安全性,因此我們試著將此技術應用於量子通訊上,希望此方法同樣可以增加量子通訊的安全性。實驗上會以高頻的隨機訊號對 SPDC (spontaneous parametric down-conversion) 產生之窄頻單光子的波包進行相位調製,使其頻寬由 4.5 MHz 展至 10 GHz,讓單光子能免於被躍遷頻率同其頻率的原子吸收或偵測,達到隱形斗篷的效果,可增加光子在傳輸過程中的隱匿性,提升量子傳輸與量子密鑰分發之安全性。

未經調製的單光子在原子的吸收頻率上時會幾乎 100\% 被吸收,實驗上我們將光子的頻寬從 2 GHz 展至 10 GHz,可讓吸收率從 70\% 降至 30\%,從此結果可了解,光子的頻寬越高能有越好的隱匿性,而提升安全性與傳輸效率。

\vspace{2em}
% \noindent \textbf{關鍵字:} \keywordsZh{} % Set keywords in config.tex
\end{document}