\documentclass[class=NCU_thesis, crop=false]{standalone}
\begin{document}

\chapter{實驗儀器與優化流程}
\section{儀器介紹}
\subsection{隨機訊號產生器}
由於實驗上無法產生真正的隨機訊號,只能使用偽隨機訊號產生器 (Pseudo Random Bit Sequence, PRBS),儀器型號為 Anritsu 的 MP1763C,可以產生 0.5 至 12.5 Gb/s 的訊號。偽隨機訊號實際上為週期訊號,會重複出現特定的隨機序列,其週期可以調整,為了達到最接近隨機的效果,我們選擇使用最長的隨機序列,一個週期內共有 $2^{31}-1$ 的隨機位元。

我們實驗上實際使用的頻率為 10 GHz (或 10 Gb/s),每秒能產生 $10\times 10^{9}$ 個隨機位元,以示波器去測量該訊號的眼圖 (eye diagram) 則可以知道訊號的品質,量測結果如下:

\fig[0.5][fig:label][!htb]{temp.png}[隨機訊號眼圖]

可見實際訊號與理論(圖)有蠻大的差異,有著相對大的上升與下降時間,圖形上下也不太對稱,這都會影響到展頻與壓縮的效果,造成實驗與理論的誤差。

\subsection{電光調製器}
電光調製器 (Electro-Optic Modulator, EOM) 可使用電訊號對光進行調製,一般而言可以分成三種,分別為振幅、相位與偏振的調製,在我們的實驗中需要調製的是相位。使用的儀器為 EOSPACE 的 SN73717 與 SN73718,分別為頻譜的窄寬與壓縮用。

相位調制器由鈮酸鋰 ($LiNbO_{3}$) 雙折射晶體製成,因泡克耳斯效應 (Pockels effect),外加電場能線性的改變快軸上的折射率,進而達到改變相位的效果,且我們稱能將 45 度線偏旋轉至 -45 度的電壓為 $V_{\pi}$。

由上介紹可知,實際使用上需優化進光的偏振以及電訊號的振幅,以達到預期的相位調製效果。

我們使用半波片 (half-wave plate) 調整入射 EOM 偏振的方向,若偏振方向不對的話,調製效果會不佳,如圖,所以實驗上優化的方式為,看著調製後的頻譜,將偏振旋轉到最接近理論模擬時的角度。

\fig[0.5][fig:label][!htb]{temp.png}[偏振角度不對]

\subsection{高頻電訊號放大器}
由於我們使用的隨機訊號產生器僅能輸出 0.2 至 2 $V_{pp}$ 的訊號,EOM 的 $V_{\pi}$ 為 2.3 V,需再經過放大器才能提供足夠的電壓去進行相位調製。
同樣的,也用示波器去測量眼圖,看放大後的訊號品質,如下圖

\fig[0.5][fig:label][!htb]{temp.png}[放大後的隨機訊號眼圖]

由於兩台放大器連接 EOM 使用的 SMA 線的材質與長短不同,會有不一樣的頻率響應與耗損,使兩個訊號無法互補,這會對頻譜壓縮與還原的效果造成負面的影響。

\subsection{法布立-培若干涉儀}
古典光可以用法布立-培若 (Fabry-Perot) 干涉儀來掃出頻譜,我們使用的儀器為 THORLABS 的(型號),FSR 為 10 GHz。
此干涉儀的主體為一個共振腔,由兩面高反射率的鏡子所組成。當光垂直入射腔體時,須滿足以下共振條件的光才能會有建設性干涉,能透射共振腔:

\begin{equation}
    2nL=m\lambda
\end{equation}
n 為共振腔的折射率,L 為腔長,頻率與透射率做圖,其中 $\nu_{F}$ 稱為 FSR (Free Spectrual Range),此參數決定了這個干涉儀適用的掃頻範圍,調整腔長 L 的大小能改變允許透射的頻率,所以若在其中一面鏡子黏上 Piezo ,輸入電壓即可微調腔長,達到掃頻的效果。

\fig[0.5][fig:label][!htb]{temp.png}[Fabry-Perot 干涉儀透射頻率]
此外,另一個重要的參數為 F (Finesse),為精細度,定義如下:

\begin{equation}
    F=\frac{\pi R^{1/2}}{1-R}
\end{equation}
此共振腔的頻寬(解析度) $\delta \lambda$ 與 F 成反比,關係如下式,所以鏡面反射率越高,F越大,解析度越好,此次實驗使用的干涉儀解析度約為 30 MHz。

\begin{equation}
    \delta \lambda=\frac{\nu_{F}}{F}
\end{equation}

\subsection{Etalon 干涉儀}
與 Fabry-Perot 干涉儀為相同的原理,只是共振腔使用的鏡子反射率較低,所以頻寬較大(約為 60 MHz),若固定腔長 L,則可做為濾波器使用,僅讓頻率寬度在 60MHz 這區間內的光通過,中心頻率則 $\nu$ 可以由溫度T改變腔長 $L(T)$ 來調整。
\todo[inline]{補上型號,確定共振腔的物質,與偏振的關係}



\end{document}