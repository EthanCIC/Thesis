\documentclass[class=NCU_thesis, crop=false]{standalone}
\begin{document}

\chapter{實驗儀器與優化流程}
本章會簡單介紹實驗上會用到的關鍵儀器,說明其特性與相關設定,並描述元件使用的優化方式與可能造成誤差之原因。

\section{隨機訊號產生器}
由於實驗上無法產生真正的隨機訊號,只能使用偽隨機訊號產生器 (Pseudo Random Bit Sequence, PRBS),儀器型號為 MP1763C (Anritsu),可以產生 0.5 至 12.5 Gb/s 的偽隨機訊號。偽隨機訊號實際上為週期訊號,會重複出現特定的隨機序列,其週期可以調整,為了達到最接近隨機的效果,我們選擇使用最長的隨機序列,一個週期內共有 $2^{31}-1$ 的隨機位元。

我們實驗上實際使用的偽隨機訊號產生率 10 Gb/s,每秒能產生 $10\times 10^{9}$ 個隨機位元,以示波器(Infiniium DCA-J 86100C, Agilent,使用的模組為 86112A, Agilent)去測量該訊號的眼圖 (eye diagram) 可以看出訊號的品質,量測結果如\cref{fig:prbs_eye},可見實際訊號與理論\cref{fig:PRBS_simulation} 有很大的差異,實際的訊號會有不小的上升與下降時間,圖形的上下也不太對稱,這都會影響到展頻與壓縮的效果,造成實驗與理論的誤差。

\begin{figure}[!hbt]
    %\captionsetup[subfigure]{labelformat=empty} % 完全隱藏圖號
    \centering
    \subcaptionbox
        {輸入第一台 EOM 之隨機訊號
        \label{fig:subfig_fig1}}
        {\includegraphics[width=0.4\linewidth]{data.bmp}}
    ~~~~
    \subcaptionbox
        {輸入第二台 EOM 之隨機訊號
        \label{fig:subfig_fig2}}
        {\includegraphics[width=0.4\linewidth]{data_bar.bmp}}
    \caption{PRBS 輸出之訊號眼圖(放大前)}
    \label{fig:prbs_eye}
\end{figure}

\section{電光調製器}

電光調製器 (Electro-Optic Modulator, EOM) 可使用電訊號對光進行調製,一般而言可以分成三種,分別為振幅、相位與偏振的調製,在我們的實驗中需要調製的是相位。使用的儀器為 EOSPACE 的 SN73717 與 SN73718,分別為頻譜的展寬與壓縮用,裝置如\cref{fig:eom}。

\fig[0.75][fig:eom][!htb]{temp.png}[實驗使用之兩台電光調製器]
\todo[inline]{補上 EOM 的照片}

相位調制器由鈮酸鋰 ($LiNbO_{3}$) 雙折射晶體製成,因泡克耳斯效應 (Pockels effect),外加電場能線性的改變作用方向上之晶體折射率,進而達到改變相位的效果,我們定義能入射光相位改變 $\pi$ 之電訊號電壓為 $V_{\pi}$。

由上介紹可知,實際使用上需優化進光的偏振以及電訊號的振幅,以達到預期的相位調製效果。所以我們會在 EOM 前放置一個偏極片(Polarizer)半波片 (half-wave plate) ,如\cref{fig:eom_setup},藉由調整兩者的角度,使光能以最佳的線偏角度入射 EOM。

\fig[0.75][fig:eom_setup][!htb]{eom_setup.png}[EOM 及相關元件架設。使用上我們會先讓光通過一偏極片,確定光為線偏振,再以半波片旋轉偏振的角度,優化調製之效果。][EOM 及相關元件架設]

\section{高頻電訊號放大器}

由於我們使用的隨機訊號產生器僅能輸出 0.2 至 2 $V_{pp}$ 的訊號,而 EOM 的 $V_{\pi}$ 高於 2 V,所以需再經過放大器才能提供足夠的電壓去進行相位調製。

我們使用的放大器型號為 OA3MVM (Centallax),輸入的訊號會經過三階段的放大,每一階段各需要兩個電壓去驅動,分別為 $V_{g}$ 與 $V_{d}$,這組電壓的大小會影響放大的大小與速度,因此我們做了一個穩壓電路,能一次輸出 3 個 $V_{g}$ 與 3 個 $V_{d}$,裝置如\cref{fig:amplider_machine}。

\fig[0.75][fig:amplider_machine][!htb]{temp.png}[放大器與穩壓電路]

同樣的,也用示波器去測量眼圖,架設如\cref{fig:amplifier_setup},觀察經放大後的訊號品質,如\cref{fig:amp_prbs_eye},可明顯看出訊號變得更不穩定,且兩台 EOM 使用的電訊號形狀也不同,這是由兩邊使用的 SMA 線的材質與長度均不同,會有不一樣的頻率響應與耗損,使兩個訊號無法互補,這會對頻譜壓縮與還原的效果造成負面的影響。

\fig[0.75][fig:amplifier_setup][!htb]{amplifier_setup.png}[將隨機訊號放大後以示波器測量眼圖]

\begin{figure}[!hbt]
    %\captionsetup[subfigure]{labelformat=empty} % 完全隱藏圖號
    \centering
    \subcaptionbox
        {輸入第一台 EOM 之隨機訊號
        \label{fig:subfig_fig1}}
        {\includegraphics[width=0.4\linewidth]{amp_data.bmp}}
    ~~~~
    \subcaptionbox
        {輸入第二台 EOM 之隨機訊號
        \label{fig:subfig_fig2}}
        {\includegraphics[width=0.4\linewidth]{amp_data_bar.bmp}}
    \caption{PRBS 輸出之訊號眼圖(放大後)}
    \label{fig:amp_prbs_eye}
\end{figure}

我們可從眼圖的測量結果去量化訊號品質,放大前後的比較如表,可見經過放大器與額外的電線連接後的訊號,波形會有所改變且穩定度會下降。

\section{法布立-培若干涉儀}
雷射光的頻譜可以用掃描式法布立-培若干涉儀(Scanning Fabry-Perot Interferometer)掃出,我們使用的儀器為 SA210-5B (THORLABS),FSR 為 10 GHz,實際儀器如\cref{fig:fabry_perot}。

\fig[0.5][fig:fabry_perot][!htb]{fabry_perot.jpg}[實驗使用之 Fabry-Perot 干涉儀]

此干涉儀的主體為一個共振腔,由兩個高反射率的凹面鏡所組成。當光正向入射腔體時,須滿足\cref{eq:resonant_condition} 之共振條件的光才會產生建設性干涉,而能透射共振腔。

\begin{equation}
    4nL=m\lambda
    \label{eq:resonant_condition}
\end{equation}
n 為共振腔的折射率,L 為腔長,頻率與透射率的關係\cref{fig:fsr},其中 $\nu_{F}$ 稱為 FSR (Free Spectrual Range),定義如\cref{eq:fsr},此參數決定了這個干涉儀適用的掃頻範圍,調整腔長 L 的長度能改變允許透射的頻率,所以若在其中一面鏡子黏上 Piezo ,輸入電壓即可微調腔長,改變允許出光的頻率,達到掃頻的效果。
\todo[inline]{確認公式}

\fig[1][fig:fsr][!htb]{fsr.png}[Fabry-Perot 干涉儀透射頻率]

\begin{equation}
    \nu_{F}=\frac{c}{4nd}
    \label{eq:fsr}
\end{equation}
此外,另一個重要的參數為精細度 F (Finesse),定義如\cref{eq:finesse_definition}:

\begin{equation}
    F=\frac{\pi R^{1/2}}{1-R}
    \label{eq:finesse_definition}
\end{equation}
此共振腔的頻寬(解析度) $\delta \lambda$ 與 F 成反比,關係如\cref{eq:resolution},所以鏡面反射率越高,F越大,解析度越好,本次實驗使用的干涉儀解析度約為 60 MHz,以頻寬約 1 MHz 掃頻結果如\cref{fig:fabry_perot_demo}。

\begin{equation}
    \delta \lambda=\frac{\nu_{F}}{F}
    \label{eq:resolution}
\end{equation}

\fig[0.75][fig:fabry_perot_demo][!htb]{temp.png}[以此 Fabry-Perot 掃出之窄頻雷射頻譜]
\todo[inline]{雷射頻譜}

\section{Etalon 干涉儀}
與 Fabry-Perot 干涉儀為相同的原理,但 Fabry-Perot 干涉儀的共振腔體為自由空間 (free space),而 Etalon 干涉儀的共振腔體則為一塊雙折射晶體,兩端為布拉格光柵結構,用以反射光形成共振腔,我們可以用 TEC 和溫控器,精準的調控晶體溫度 T 改變腔長 $L(T)$,只讓特定中心頻率 $\nu$ 附近的光通過。我們實驗使用的型號為 AF023G (MICRO OPTICS, INC.),頻寬為 60 HMz,FSR 為 13.6 GHz,裝置如\cref{fig:etalon}。

\fig[0.75][fig:etalon][!htb]{etalon.png}[Etalon 濾波器裝置圖]

由於腔體是由雙折射晶體製成,所以不同的偏振在內部會有不一樣的行進速度,而會產生兩組不同的模態,所以在實驗優化上,需要將入射 Etalon 干涉儀的光調成與晶軸方向相同的線偏光,才能最有效率的使用濾波器。

\end{document}