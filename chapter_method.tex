\documentclass[class=NCU_thesis, crop=false]{standalone}
\begin{document}

\chapter{實驗方法與架設}
\section{儀器介紹}
\subsection{隨機訊號產生器}
由於實驗上無法產生真正的隨機訊號,只能使用偽隨機訊號產生器 (Pseudo Random Bit Sequence, PRBS),儀器型號為 Anritsu 的 MP1763C,可以產生 0.5 至 12.5 GHz 的訊號,偽訊號的週期可以調整,為了達到最接近隨機的效果,我們選擇使用最長的隨機序列,一組共有 $2^{31}-1$ 的位元。

我們實驗上實際使用的頻率為 10 GHz 或 10 Gb/s,每秒能產生 $10\times 10^{9}$ 個隨機位元,以示波器去測量該訊號的眼圖 (eye diagram) 則可以知道訊號的品質,量測結果如下:

可見實際訊號與理論(圖)有蠻大的差異,有著相對大的上升與下降時間,圖形上下也不太對稱,這都會影響到展頻與壓縮的效果,造成實驗與理論的誤差。

\subsection{電光調製器}
電光調製器可使用電訊號對光進行調製,一般而言可以分成三種,分別為振幅、相位與偏振的調製,在我們的實驗中需要調製的是相位。使用的儀器為 EOSPACE 的 SN73717 與 SN73718,分別為頻譜的窄寬與壓縮用。

相位調制器由鈮酸鋰 ($LiNbO_{3}$) 雙折射晶體製成,因泡克耳斯效應 (Pockels effect),外加電場能線性的改變快軸上的折射率,進而達到改變相位的效果,且我們稱能將 45 度線偏旋轉至 -45 度的電壓為 $V_{\pi}$。

由上介紹可知,實際使用上需優化進光的偏振以及電訊號的振幅,以達到預期的相位調製效果。

\subsection{高頻電訊號放大器}
由於我們使用的隨機訊號產生器僅能輸出 0.2 至 2 $V_{pp}$ 的訊號,EOM 的 $V_{\pi}$ 為 2.3 V,需再經過放大器才能提供足夠的電壓去進行相位調製。
同樣的,也用示波器去測量眼圖,看放大後的訊號品質,如下圖

由於兩台使用的 SMA 線的材質與長短不同,會有不一樣的頻率響應與耗損,使兩個訊號無法互補,這會影響頻譜壓縮的品質。

\subsection{Fabry-Perot 干涉儀}
古典光可以用 Fabry-Perot 干涉儀來掃出頻譜,我們使用的儀器為 THORLABS 的,FSR 為 10 GHz。
此干涉儀為一個共振腔,由兩面高反射率的鏡子所組成
\todo[inline]{補上型號與重要參數和示意圖}

\subsection{Etalon 干涉儀}
與 Fabry-Perot 干涉儀為相同的原理,只是共振腔使用的鏡子反射率較低,所以線寬較大,約為 60 MHz,若固定腔長 L,則可做為濾波器使用,僅讓頻率寬度在 60MHz 這區間內的光通過,中心頻率則可以由溫度T改變腔長 $L(T)$ 來調整。
\todo[inline]{補上型號,確定共振腔的物質,與偏振的關係}

\section{單光子光源製備}
單光子的產生機制為 SPDC,使用 PPKTP 晶體產生 Type-II 的時間 - 能量糾纏光子對。

調整 pump 光與晶體溫度來改變單光子的頻率
\todo[inline]{G2簡介與放上G2圖}

\section{光路架設}
\subsection{古典光量測}
古典光
時間差很重要!
\subsection{量子光量測}
由於量子光無法用 Fabry-Perot 掃頻譜,所以要在光路的最後加上 Etalon,只允許 60MHz 內的光通過,以確保第二台相位調製器有將展寬的頻譜壓縮回窄頻。

\end{document}