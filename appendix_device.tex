\documentclass[class=NCU_thesis, crop=false]{standalone}
\begin{document}

\chapter{裝置列表}


\section{法布立-培若干涉儀}
雷射光的頻譜可以用掃描式法布立-培若干涉儀(Scanning Fabry-Perot Interferometer)掃出,我們使用的儀器為 SA210-5B (THORLABS),FSR 為 10 GHz,實際儀器如\cref{fig:fabry_perot}。

\fig[0.5][fig:fabry_perot][!htb]{fabry_perot.jpg}[實驗使用之 Fabry-Perot 干涉儀]

此干涉儀的主體為一個共振腔,由兩個高反射率的凹面鏡所組成。當光正向入射腔體時,須滿足\cref{eq:resonant_condition} 之共振條件的光才會產生建設性干涉,而能透射共振腔。

\begin{equation}
    4nL=m\lambda
    \label{eq:resonant_condition}
\end{equation}
n 為共振腔的折射率,L 為腔長,頻率與透射率的關係\cref{fig:fsr},其中 $\nu_{F}$ 稱為 FSR (Free Spectrual Range),定義如\cref{eq:fsr},此參數決定了這個干涉儀適用的掃頻範圍,調整腔長 L 的長度能改變允許透射的頻率,所以若在其中一面鏡子黏上 Piezo ,輸入電壓即可微調腔長,改變允許出光的頻率,達到掃頻的效果。

\fig[1][fig:fsr][!htb]{fsr.png}[Fabry-Perot 干涉儀透射頻率]

\begin{equation}
    \nu_{F}=\frac{c}{4nd}
    \label{eq:fsr}
\end{equation}
此外,另一個重要的參數為精細度 F (Finesse),定義如\cref{eq:finesse_definition}:

\begin{equation}
    F=\frac{\pi R^{1/2}}{1-R}
    \label{eq:finesse_definition}
\end{equation}
此共振腔的頻寬(解析度) $\delta \lambda$ 與 F 成反比,關係如\cref{eq:resolution},所以鏡面反射率越高,F越大,解析度越好。

\begin{equation}
    \delta \lambda=\frac{\nu_{F}}{F}
    \label{eq:resolution}
\end{equation}

為了知道此 Fabry-Perot 干涉儀的頻寬,我們在\cref{fig:laser_bandwidth_setup} 的光路架設下,以 Fabry-Perot 干涉儀對我們的窄頻雷射(頻寬約 1 MHz)掃頻。使用時要先調整輸入 Piezo 的週期訊號的電壓大小,直到能在一個振盪週期內看到兩個訊號為止,測量結果如\cref{fig:FSR_laser},此時兩個訊號的間距即為一個 FSR,也就是 10 GHz。但以示波器(DPO4104B, Tektronix)測得的頻譜橫軸為時間(單位為秒),我們可從測量結果求得間與頻率之對應關係(0.8459 秒對應 10 GHz)。接著放大其中一個訊號,測量結果如圖\cref{fig:laser_bandwidth},其半高全寬(Full Width at Half Maximum, FWHM)的時間寬度為 0.000498 秒,利用上述之對應關係,即可算出此 Fabry-Perot 干涉儀之頻寬為 58 MHz。

\fig[0.75][fig:laser_bandwidth_setup][!htb]{laser_bandwidth_setup.png}[Fabry-Perot 干涉儀頻寬測量架設圖]
\fig[0.6][fig:FSR_laser][!htb]{FSR_laser.png}[Fabry-Perot 干涉儀 FSR 測量]
\fig[0.6][fig:laser_bandwidth][!htb]{laser_bandwidth.png}[Fabry-Perot 干涉儀頻寬測量]

\clearpage
\section{Etalon 干涉儀}
\label{section:etalon}
與 Fabry-Perot 干涉儀為相同的原理,但 Fabry-Perot 干涉儀的共振腔體為自由空間 (free space),而 Etalon 干涉儀的共振腔體則為一塊雙折射晶體,兩端為布拉格光柵結構,用以反射光形成共振腔,我們可以用 TEC 和溫控器,精準的調控晶體溫度 T 改變腔長 $L(T)$,只讓特定中心頻率 $\nu$ 附近的光通過。我們實驗使用的型號為 AF023G (MICRO OPTICS, INC.),頻寬為 60 HMz,FSR 為 13.6 GHz,裝置如\cref{fig:etalon}。

\fig[0.75][fig:etalon][!htb]{etalon.png}[Etalon 濾波器裝置圖]

由於腔體是由雙折射晶體製成,所以不同的偏振在內部會有不一樣的行進速度,而會產生兩組不同的模態,所以在實驗優化上,需要將入射 Etalon 干涉儀的光調成與晶軸方向相同的線偏光,才能最有效率的使用濾波器。


\end{document}