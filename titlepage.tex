\documentclass[class=NTHU_thesis, crop=false, float=true]{standalone}
\begin{document}
% I use LaTeX3 to automatically generate name table. 
% Below \ExplSyntaxOn to \ExplSyntaxOff perpare prof. table contents,
% it will save contents to `\profsTableContent''. 
% You can ignore this block even you want make table by yourself.
\ExplSyntaxOn
% Copy prof. list from config.tex
\clist_gclear_new:N \g_sppmg_profsZh_cl
\clist_gset:NV \g_sppmg_profsZh_cl \profsZh
\clist_gclear_new:N \g_sppmg_profsEn_cl
\clist_gset:NV \g_sppmg_profsEn_cl \profsEn

% get total number of prof. . Omitted language will not display.
\int_gzero_new:N \g_sppmg_profTotal_int
\int_gset:Nn \g_sppmg_profTotal_int {
    \int_max:nn {\clist_count:N \g_sppmg_profsZh_cl} 
        {\clist_count:N \g_sppmg_profsEn_cl}
}

% NOTE: ``tabularx'' will  processes its contents more than once 
% for calculate width, so ``gpop'' can't put in tabularx env.
% \tl_if_exist:NTF {\tl_clear:N \g_sppmg_tableContent_tl} {\tl_new:N \g_sppmg_tableContent_tl}
% \tl_if_exist:NTF \g_sppmg_tableContent_tl {} {\tl_new:N \g_sppmg_tableContent_tl}
\tl_gclear_new:N \g_sppmg_tableContent_tl


% Use a inline function for pop 2 list (zh+en), and save table content 
% Input(#1) switch 3 case, 1 = Advisor, 2 = committee member , 3+ is more.
% Use ``for'' loop to get all prof.
\int_step_inline:nnnn {1}{1}{\g_sppmg_profTotal_int}{
    \clist_gpop:NNTF \g_sppmg_profsZh_cl \l_tmpa_tl {}{ \tl_clear:N \l_tmpa_tl}
    \clist_gpop:NNTF \g_sppmg_profsEn_cl \l_tmpb_tl {}{ \tl_clear:N \l_tmpb_tl}
    \tl_gput_right:Nx \g_sppmg_tableContent_tl {
        \int_case:nnTF {#1}{
            {1} {指導教授: & \l_tmpa_tl & 博士 & (Dr.~ \l_tmpb_tl ) \exp_not:n {\\} }
        }{}{
            & \l_tmpa_tl & 教授 & (Prof.~ \l_tmpb_tl ) \exp_not:n {\\} 
        }
    }
}

% Copy contents to LaTeX2e macro.
\cs_set_eq:NN \profsTableContent \g_sppmg_tableContent_tl

\ExplSyntaxOff
\def\fsUniversity{\fs{24}[1.5]}
\def\fsTitle{\fs{20}[1.5]}
\def\fsNames{\fs{18}[1.5]}

\def\mcShift{\hspace{-6.0pt}} % It's for align non-multicolumn cell .
% --------define title page layout for thesis
\titlepageFontFamily % set in config.tex
\newgeometry{top=2.5cm, bottom=2.5cm, inner=2cm, outer=2cm} % only for titlepage
\begin{spacing}{1.0}
\begin{titlepage}
    \null\vfill
    \begin{center}
        {\fsUniversity 國\quad 立\quad 清\quad 華\quad 大\quad 學} \par
        \vspace*{5mm}
        
        {\fsTitle {\degree}論文\par}
        \vspace*{20mm}
        
        {\fsTitle {\title} \par}
        \vspace*{5mm}
        
        {\fsTitle {\subtitle} \par}
        \vspace*{10mm}
        
        {\ifx \logo\empty\vspace*{30mm}
        \else \includegraphics[height=30mm]{\logo} \par
        \fi}
        \vfill
        
        {\fsNames \renewcommand{\arraystretch}{1}
            % If you want make table by yourself, replace ``\profsTableContent''
            \begin{tabular}{l@{\hspace*{0.4em}}l@{\quad}l@{\quad}l}
            系\qquad 所: & \multicolumn{3}{l}{\mcShift \dept}      \\
            學\qquad 號: & \multicolumn{3}{l}{\mcShift \studentID} \\
            研\enspace 究\enspace 生: & \authorZh & & (\authorEn)    \\
            
            \profsTableContent
            
            \end{tabular}
        \par}
        \vspace*{5ex}
        
        {\fsTitle {\degreedate} \par}
        \vspace*{2ex}
        
        \ifthenelse{\boolean{printcopyright}}
        {{{版權所有\copyright\ \author\ \quad \copyyear} \par}}
        {}
    \end{center}
    \null\vfill
\end{titlepage}
\end{spacing}
\restoregeometry
\normalfont % use main font
%--------end of title page for thesis
\cleardoublepage
\end{document}
