\documentclass[class=NCU_thesis, crop=false]{standalone}
\usepackage{mathrsfs,amsmath}
\begin{document}

\chapter{基本原理介紹}
實現產頻技術的方式有很多種,本章會介紹展頻技術的基本原理,說明如何以相位調製去實作直接序列展頻 (DSSS)。會從數學原理出發,瞭解隨時間調變相位會如何改變頻率的分佈,再以單頻波之相位調製去模擬,觀察調製前與調製後頻譜的差異。

\section{展頻技術}

展頻技術 (spread spectrum technology) 是一種可將原訊號的頻譜打散分佈到比原始頻寬更寬的技術。我們的實驗選用的方法為直接序列展頻,能將窄頻雷射 (narrow-band laser) 與單光子的頻寬從約 10 MHz 展至 10 GHz,其作法為,以 PRBS 產生高頻隨機訊號,使用光電調製器 (EOM) 對入射光進行相位調製,此在時域上的操作,經傅立葉轉換後等效於增加其他頻率成分,以達到展寬頻率的效果。
\section{相位調製}

\subsection{數學形式}
此小節介紹相位調製的數學形式。設入射 EOM 的雷射波函數為 $E_{0}(t)$,調製函數 (modulation function) 為 $M(t)$,經調製後的波函數 $E_{m}(t)$ 可表示成:
\begin{equation}
    E_{m}(t)=E_{0}(t)e^{iM(t)}
\end{equation}
若對此式做傅立葉轉換,根據 convolution theorem,可得:
\begin{equation}
\label{eq:modulation_function}
    \mathscr{F}\{E_{0}(t)e^{iM(t)}\}=\tilde{E_{0}}(\omega)*\mathscr{F}\{{e^{iM(t)}}\}
\end{equation}
$\tilde{E_{0}}(\omega)$ 為入射光之頻譜,所以在頻譜數學分析上,我們可以把入射光頻譜與相位調製的部分分開處理,個別將兩項計算好後再做摺積即可得到調製後的頻譜。

\subsection{單頻波}
若入射光的頻譜為中心頻率在 $\nu_{0}$ 的勞倫茲分佈(lorenz distribution),調製函數為頻率 $\nu_{m}$ 的單頻波,意即輸入的電訊號強度隨時間的函數可表示為 $\phi_{0} sin(2\pi \nu_{m} \omega t)$,則可將\cref{eq:modulation_function} 改寫為:
\begin{equation}
\label{eq:convolution_theorem}
    \mathscr{F}\{E_{0}(t)e^{i\phi_{0} sin(2\pi \nu_{m} \omega t)}\}=\tilde{E_{0}}(\omega)*\mathscr{F}\{{e^{i\phi_{0} sin(2\pi \nu_{m} \omega t)}}\}
\end{equation}
其中 $\tilde{E_{0}}(\omega)$ 為勞倫茲分佈,另一項傅立葉轉換的結果為第一類貝索函數(Bessel function of the first kind):
\begin{equation}
    \mathscr{F}\{{e^{i\phi_{0} sin(2\pi \nu_{m} \omega t)}}\}=J_{n}(\phi_{0})
\end{equation}
或在時域上看,將調製項做傅立葉級數展開:
\begin{equation}
    e^{i\phi_{0} sin(2\pi \nu_{m} \omega t)}=\sum_{n=-\infty}^{\infty}J_{n}(\phi_{0})e^{i 2 \pi n \nu_{m} t}
\end{equation}
可從上式看出,調製項的頻譜是由頻率為 $n \nu_{m}$ 的狄拉克函數(Dirac function) 組成,$n=0, \pm1, \pm2, ...$,強度分佈為 $J_{n}(\phi_{0})$,如\cref{fig:bessel_function},可看出在不同的調製電壓(也就是不同的$\phi_{0}$)時會有不一樣強度的頻率分布,以 $\phi_{0}=0.75\pi$ 為例,模擬對一道頻寬為 60 MHz 的雷射做 1 GHz 的正弦波相位調製,從\cref{eq:convolution_theorem} 可知,將入射光(\cref{fig:monocromatic_modulation} 黑線)與調製項的頻譜做摺積可得調製後的結果(\cref{fig:monocromatic_modulation} 紅線),每個頻率的間隔為 1 GHz。兩者比較可明顯看出,時域上相位調製能讓改變頻率的分佈。

\fig[0.75][fig:bessel_function][!htb]{bessel_function.png}[貝索函數]

以 $\phi_{0}=0.75\pi$ 為例,模擬對一道頻寬為 60 MHz 的雷射做 1 GHz 的正弦波相位調製,從\cref{eq:convolution_theorem} 可知,將入射光(\cref{fig:monocromatic_modulation} 黑線)與調製項的頻譜做摺積可得調製後的結果(\cref{fig:monocromatic_modulation} 紅線),且每個頻率的間隔為 1 GHz。兩者比較可明顯看出,時域上相位調製能讓改變頻率的分佈。

\fig[0.75][fig:monocromatic_modulation][!htb]{monocromatic_modulation.png}[單頻波相位調製。黑線為調製前;紅線為調製後。][單頻波相位調製]

\end{document}