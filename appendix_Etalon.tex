\documentclass[class=NCU_thesis, crop=false]{standalone}
\begin{document}

\chapter{Etalon 干涉儀}
\label{chaper:etalon}
與 Fabry-Perot 干涉儀為相同的原理,但 Fabry-Perot 干涉儀的共振腔體為自由空間 (free space),而 Etalon 干涉儀的共振腔體則為一塊雙折射晶體,兩端為布拉格光柵結構,用以反射光形成共振腔,我們可以用 TEC 和溫控器,精準的調控晶體溫度 T 改變腔長 $L(T)$,只讓特定中心頻率 $\nu$ 附近的光通過。我們實驗使用的型號為 AF023G (MICRO OPTICS, INC.),頻寬為 60 HMz,FSR 為 13.6 GHz,裝置如\cref{fig:etalon}。

\fig[0.75][fig:etalon][!htb]{etalon.png}[Etalon 濾波器裝置圖]

由於腔體是由雙折射晶體製成,所以不同的偏振在內部會有不一樣的行進速度,而會產生兩組不同的模態,所以在實驗優化上,需要將入射 Etalon 干涉儀的光調成與晶軸方向相同的線偏光,才能最有效率的使用濾波器。

\end{document}