\documentclass[class=NCU_thesis, crop=false]{standalone}
\begin{document}

\chapter{總結}
我們示範了如何將展頻技術運用於量子通訊,能以更隱密的方式傳輸光子,讓單光子免於被躍遷頻率同其頻率的原子吸收或偵測,可以提升量子密鑰分發 (Quantum Key Distribution) 與量子多工技術 (Qauntum Multiplexing) 在傳輸過程中之安全性。

實驗上我們透過調控雙光子波包之相位來改變光子的頻寬,當頻寬被展至越寬時,光子與原子之交互作用越小。除了單光子以外,我們還用雷射光進行相同的量測來確定理論的正確性與一致性。從實驗結果可知,以越高速的隨機訊號對光子進行相位調製,越能提升光子的隱匿性,且我們能從實驗的誤差去修正理論模擬,了解在實際上儀器的品質如何影響調製的效果。

在未來的應用上,雖然使用高速的訊號對光子做相位調製能有越好的隱匿性,但因受限於儀器,包括線材與訊號產生器,越快的訊號越不穩定,會越難將訊號還原成初始的狀態,所以必須根據實際的需求與設備,在安全性與資料傳輸效率或訊號品質之間取捨。

\end{document}