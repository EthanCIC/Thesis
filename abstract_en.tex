\documentclass[class=NCU_thesis, crop=false]{standalone}
\begin{document}

\chapter{Abstract}
In classical communication, spread spectrum technology is an effective way to improve the security in transmission process. We try to implement this technology in quantum communication as well. In our experiment, we have used synchronized high-speed electro- optic modulators to modulate the phase of resonant single photon and biphoton wave packets, which are generated from cavity enhanced SPDC (spontaneous parametric down-conversion) and broaden their bandwidths from 4.5 MHz to 10 GHz for avoiding photons being absorbed or detected by the atoms. The reduced absorption is like imposing an invisible cloak over a photon, improving the confidentiality during the transmission process in order to improve the security of quantum communication and key distribution. 

In principle, atoms will absorb on-resonant narrowband (smaller than natural linewidth) single photons in our experiment. we broadened the bandwidth of the photon to a wide range between 2~10 GHz, with the absorption rate decreasing to 70~30\% By these results, our work shows that, for transmitted photons, the higher the bandwidth, the higher the confidentiality.

\vspace{2em}
% \noindent \textbf{Keywords:} \keywordsEn{} % Set keywords in config.tex
\end{document}